\documentclass[11pt]{article}

\usepackage{amssymb}
\usepackage{amsmath}
\usepackage{graphicx}
\usepackage{hyperref}

\def\N{{\mathcal N}}
\def\R{{\mathcal R}}
\def\E{{\mathbb E}}

\setlength{\oddsidemargin}{0.25 in}
\setlength{\evensidemargin}{-0.25 in}
\setlength{\topmargin}{-0.6 in}
\setlength{\textwidth}{6.5 in}
\setlength{\textheight}{8.5 in}
\setlength{\headsep}{0.75 in}
\setlength{\parindent}{0.25 in}
\setlength{\parskip}{0.1 in}

\newcommand{\lecture}[4]{
   \pagestyle{myheadings}
   \thispagestyle{plain}
   \newpage
   \setcounter{page}{1}
   \setcounter{section}{0}
   \noindent
   \begin{center}
   \framebox{
      \vbox{\vspace{2mm}
    \hbox to 6.28in { {\bf Statistical Learning \hfill #4} }
       \vspace{6mm}
       \hbox to 6.28in { {\Large \hfill #1  \hfill} }
       \vspace{6mm}
       \hbox to 6.28in { {\it Instructor: #2\hfill } }
      \vspace{2mm}}
   }
   \end{center}
   \markboth{#1}{#1}
   \vspace*{4mm}
}


\begin{document}

\lecture{Homework 7. Support Vector Machine and Flexible Discriminant Analysis}{Yuan Yao}{}{Deadline: November 16, 2015}

% The problem below marked by $^*$ is optional with bonus credits.

For the experimental problem, include the source codes (as Appendix) which are runnable under standard settings. On the head of your submitted homework, please mark \emph{NAME - student ID}.

By (ESL), please refer to the Elements of Statistical Learning, Edition II, the 10$^{th}$ print

\url{http://statweb.stanford.edu/~tibs/ElemStatLearn/printings/ESLII_print10.pdf}

\begin{enumerate}
\item (ESL) Exercise 12.1

\item (ESL) Exercise 12.2

\item (ESL) Exercise 12.3

\item (ESL) Exercise 12.5

\item[A*] (Crowdsourced voting on Age Identification) It is always hard to tell the age given a face image, while it is relatively easy to identify which one looks older given two faces presented together. The following crowdsourcing task, created at a website made by Tsinghua University, aims to collect pairwise comparison data for age identification. If you have not done this, please try --
\subitem[1.] Go to the following website and create a \emph{worker} account \\
\url{http://www.chinacrowds.com}
\subitem[2.] Sign in with your worker account and choose task 370 for age comparisons
\subitem[3.] Input your voting (preferred at least 50 tasks for each person): 10 tasks per page and remember to submit when you finish a page.
%
%\item[*B] Explore the prostate cancer data with the following methods
%\subitem[1] Least Square method, understand the confidence intervals of parameters
%\subitem[2] Regularization path of Ridge Regression
%\subitem[3] Regularization path of LASSO
\end{enumerate}

\end{document}


\item {\em Finite rank perturbations of random symmetric matrices:} Wigner's semi-circle law (proved by Eugene Wigner in 1951) concerns the limiting distribution of the eigenvalues of random symmetric matrices. It states, for example, that the limiting eigenvalue distribution of $n\times n$ symmetric matrices whose entries $w_{ij}$ on and above the diagonal $(i\leq j)$ are i.i.d Gaussians $\mathcal{N}(0,\frac{1}{4n})$ (and the entries below the diagonal are determined by symmetrization, i.e., $w_{ji}=w_{ij}$) is the semi-circle:
    $$p(t) = \frac{2}{\pi} \sqrt{1-t^2}, \quad -1\leq t \leq 1,$$
    where the distribution is supported in the interval $[-1,1]$.
\begin{enumerate}
\item Confirm Wigner's semi-circle law using MATLAB simulations (take, e.g., $n=400$).
\item Find the largest eigenvalue of a rank-1 perturbation of a Wigner matrix. That is, find the largest eigenvalue of the matrix $$W + \lambda_0 uu^T,$$ where $W$ is an $n\times n$ random symmetric matrix as above, and $u$ is some deterministic unit-norm vector. Determine the value of $\lambda_0$ for which a phase transition occurs. What is the correlation between the top eigenvector of $W+\lambda_0 uu^T$ and the vector $u$ as a function of $\lambda_0$? Use techniques similar to the ones we used in class for analyzing finite rank perturbations of sample covariance matrices.
\end{enumerate}


