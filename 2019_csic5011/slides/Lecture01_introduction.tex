\documentclass[10pt,mathserif]{beamer}

\usepackage{graphicx,amsmath,amssymb,tikz,psfrag,hyperref}

\input defs.tex

%% formatting

\mode<presentation>
{
\usetheme{default}
}
\setbeamertemplate{navigation symbols}{}
\usecolortheme[rgb={0.13,0.28,0.59}]{structure}
\setbeamertemplate{itemize subitem}{--}
\setbeamertemplate{frametitle} {
	\begin{center}
	  {\large\bf \insertframetitle}
	\end{center}
}

\newcommand\footlineon{
  \setbeamertemplate{footline} {
    \begin{beamercolorbox}[ht=2.5ex,dp=1.125ex,leftskip=.8cm,rightskip=.6cm]{structure}
      \footnotesize \insertsection
      \hfill
      {\insertframenumber}
    \end{beamercolorbox}
    \vskip 0.45cm
  }
}
\footlineon

\AtBeginSection[] 
{ 
	\begin{frame}<beamer> 
		\frametitle{Outline} 
		\tableofcontents[currentsection,currentsubsection] 
	\end{frame} 
} 

%% begin presentation

\title{\large \bfseries Lecture 1. Introduction and Syllabus}

\author{Yuan Yao\\[3ex]
Hong Kong University of Science and Technology}

\date{\today}

\begin{document}

\frame{
\thispagestyle{empty}
\titlepage
}

\section{Geometric Data Analysis}

\begin{frame}
\frametitle{Part I. Geometric Data Analysis}
\begin{itemize}\itemsep=12pt
\item A duality in linear dimensionality reduction
\vspace*{0.5em}
\begin{itemize}
\item Principal Component Analysis (PCA)
\item Multidimensional Scaling (MDS)
\item Random matrix theory and phase transitions
\item Random projection and restricted isometry property
\end{itemize}
\vspace*{0.5em}
\item Extended PCA/MDS via SDP
\vspace*{0.5em}
\begin{itemize}
\item Robust PCA
\item Sparse PCA
\item Graph Realization or Sensor Network Localization
\end{itemize}
\item Manifold Learning: nonlinear dimensionality reduction via spectral method on graphs
\vspace*{0.5em}
\begin{itemize}
\item Locally Linear Embedding (PCA+), Isomap (MDS+)
\item Laplacian LLE, Diffusion Map, LTSA
\end{itemize}
\end{itemize}
\end{frame}

\begin{frame}
\frametitle{Part I. Geometric Data Analysis (continued)}
\begin{itemize}\itemsep=12pt
\item Supervised PCA
\vspace*{0.5em}
\begin{itemize}
\item Ridge Regression and PCA
\item Slice Inverse Regression and Linear Discriminant Analysis
\end{itemize}
\vspace*{0.5em}
\item *Other topics in representation learning
\vspace*{0.5em}
\begin{itemize}
\item $t$SNE
\item Steerable PCA
\item Dictionary learning and Matrix Factorization
\item Deep learning
\end{itemize}
\end{itemize}
\end{frame}


%\begin{frame}
%\frametitle{Pictures with tikz}
%\begin{center}
%\begin{tikzpicture}
%	[scale=1.5,dot/.style={circle,draw=black!100,fill=black!100,thick,inner sep=0pt,minimum size=2pt}]
%    \node[dot] at (-1,0) (n1) {};
%    \node[dot] at (0,1)  (n2) {};
%    \node[dot] at (1,0)  (n3) {};
%    \node[dot] at (0,-1) (n4) {};
%    \draw[gray] (-1.5,0) -- (1.5,0);
%    \draw[gray] (0,-1.5) -- (0,1.5);
%    \draw[black,thick] (n1) -- (n2) -- (n3) -- (n4) -- (n1) -- cycle;
%    \draw[orange,thick] (-1,0.5) -- (0,1) -- (1,1.5);
%\end{tikzpicture}
%\qquad
%\begin{tikzpicture}
%	[scale=1.5,dot/.style={circle,draw=black!100,fill=black!100,thick,inner sep=0pt,minimum size=2pt}]
%    \draw[gray] (-1.5,0) -- (1.5,0);
%    \draw[gray] (0,-1.5) -- (0,1.5);
%    \draw[black,thick] (-1,1) -- (0,0) -- (1,1);
%\end{tikzpicture}
%\end{center}
%\end{frame}
%
%\begin{frame}
%\frametitle{Pictures with tikz}
%\begin{itemize}\itemsep=12pt
%	\item convex envelope of (nonconvex) $f$ is the largest convex underestimator $g$
%    \item \ie, the best convex lower bound to a function
%        \vspace*{1em}
%\begin{center}
%\begin{tikzpicture}
%    \draw[gray] (-1.5,0) -- (1.5,0);
%    \draw[gray] (0,-0.5) -- (0,1.5);
%    \draw[black,thick] (-1,1) -- (0,0) -- (0.5,0.5) -- (1,-0.25) -- (2,1);
%    \draw[black,dashed] (0,0) -- (1,-0.25);
%\end{tikzpicture}
%\end{center}
%    \item \textbf{example}: $\ell_1$ is the envelope of $\card$ (on unit $\ell_\infty$ ball)
%    \item \textbf{example}: $\|\cdot\|_*$ is the envelope of $\rank$ (on unit spectral norm ball)
%    \item various characterizations: \eg, $f^{**}$ or convex hull of epigraph
%\end{itemize}
%\end{frame}

\section{Topological Data Analysis}

\begin{frame}
\frametitle{Part II. Topological Data Analysis}
\begin{itemize}\itemsep=12pt
\item Clustering method (0-homology)
\vspace*{0.5em}
\begin{itemize}
\item $k$-center
\item $k$-means
\item hierarchical linakge
\end{itemize}
\vspace*{0.5em}
\item Topological Data Analysis and Morse Theory
\vspace*{0.5em}
\begin{itemize}
\item Reeb graph and mapper
\item Persistent homology and discrete Morse theory
\item *Critical nodes and graphs
\end{itemize}
\item *Euler Calculus and signal processing
\end{itemize}
\end{frame}

\begin{frame}
\frametitle{Part I. Topological Data Analysis (continued)}
\begin{itemize}\itemsep=12pt
\item Hodge Theory: a bridge connecting geometry and topology
\vspace*{0.5em}
\begin{itemize}
\item Spectral clustering and graph Laplacian
\item Statistical ranking and graph Helmoholtzian/Hodge Laplacian
\begin{itemize}
\item Experimental design and random graph theory
\item Online ranking and stochastic algorithms
\item Budget control and information maximization
\item Individual learning vs. social choice theory
\end{itemize}
\item Game theory
\vspace*{0.5em}
\begin{itemize}
\item Finite game flow and combinatorial Hodge Theory
\item Differentiable games (GANs), stochastic games and Markov decision process (reinforcement learning)
\end{itemize}
\end{itemize}
\end{itemize}
\end{frame}

\section{Planned Schedule}

\begin{frame}
\frametitle{Planned Schedule}
\begin{itemize}\itemsep=12pt
\item The course runs for about 13 weeks.
\vspace*{0.5em}
\item Week 1: 
\vspace*{0.5em}
\begin{itemize}
\item Jan 30: Introduction 
\item Feb 1:  seminar by Ruohan ZHAN (Stanford University) with title "Safety masked reinforcement learning"
\end{itemize}
\item Week 2: spring festival break (Feb 8 will be rescheduled to later)
\vspace*{0.5em}
\item Week 3: 
\vspace*{0.5em}
\begin{itemize}
\item Feb 13: PCA
\item Feb 15: MDS
\end{itemize}
\end{itemize}
\end{frame}

\begin{frame}
\frametitle{Planned Schedule (continued)}
\begin{itemize}\itemsep=12pt
\item Feb 20 - May 8: to-be-announced on courseweb
\vspace*{0.5em}
\begin{itemize}
\item \url{https://yao-lab.github.io/2019_csic5011/}
\end{itemize}
\item Occasionally invited speakers from academia or industry will present
\vspace*{0.5em}
\item Discussions on piazza (by invitation only):
\vspace*{0.5em}
\begin{itemize}
\item \url{https://piazza.com/ust.hk/spring2019/csic5011/home}
\end{itemize}
\end{itemize}
\end{frame}


%\begin{frame}
%\frametitle{Group lasso \\[-0.3em] 
%{\footnotesize \textmd{(\eg, Yuan \& Lin; Meier, van de Geer, B\"uhlmann; Jacob, Obozinski, Vert)}}}
%\begin{itemize}\itemsep=12pt
%\item problem:
%\[
%\begin{array}{ll}
%\mbox{minimize} & f(x) + \lambda \sum_{i=1}^N \|x_i\|_2
%\end{array}
%\]
%\ie, like lasso, but require groups of variables to be zero or not
%\item also called $\ell_{1,2}$ mixed norm regularization
%\end{itemize}
%\end{frame}
%
%\begin{frame}{Structured group lasso \\[-0.3em] 
%{\footnotesize \textmd{(Jacob, Obozinski, Vert; Bach et al.; Zhao, Rocha, Yu; \dots)}}}
%\begin{itemize}\itemsep=12pt
%\item problem:
%\[
%\begin{array}{ll}
%\mbox{minimize} & f(x) + \sum_{i=1}^N \lambda_i \|x_{g_i}\|_2
%\end{array}
%\]
%where $g_i \subseteq [n]$ and $\mathcal G = \{g_1, \dots, g_N\}$
%\item like group lasso, but the groups can overlap arbitrarily
%\item particular choices of groups can impose `structured' sparsity
%\item \eg, topic models, selecting interaction terms for (graphical) models,
%    tree structure of gene networks, fMRI data
%\item generalizes to the \textbf{composite absolute penalties family}:
%\[
%r(x) = \|(\|x_{g_1}\|_{p_1}, \dots, \|x_{g_N}\|_{p_N})\|_{p_0}
%\]
%\end{itemize}
%\end{frame}
%
%\begin{frame}{Structured group lasso \\[-0.3em] 
%{\footnotesize \textmd{(Jacob, Obozinski, Vert; Bach et al.; Zhao, Rocha, Yu; \dots)}}}
%\textbf{hierarchical selection}:
%\begin{center}
%\begin{tikzpicture}
%[dot/.style={rectangle,draw=black,fill=white,inner sep=5pt,minimum size=5pt}]
%\node[dot,draw=orange,thick] at (0,5) (n1) {1};
%\node[dot] at (-1,4) (n2) {2};
%\node[dot,draw=orange,thick] at (1,4) (n3) {3};
%\node[dot] at (-1,3) (n4) {4};
%\node[dot,draw=orange,thick] at (0.5,3) (n5) {5};
%\node[dot] at (1.5,3) (n6) {6};
%\draw[->] (n1) -- (n2);
%\draw[->] (n1) -- (n3);
%\draw[->] (n2) -- (n4);
%\draw[->] (n3) -- (n5);
%\draw[->] (n3) -- (n6);
%\end{tikzpicture}
%\end{center}
%\begin{itemize}\itemsep=8pt
%    \item $\mathcal G = \{ \{4\}, \textcolor{orange}{\{5\}}, \{6\}, \{2,4\}, 
%        \textcolor{orange}{\{3,5,6\}}, \textcolor{orange}{\{1,2,3,4,5,6\} \}}$
%\item nonzero variables form a rooted and connected subtree
%    \begin{itemize}
%        \item if node is selected, so are its ancestors
%        \item if node is not selected, neither are its descendants
%    \end{itemize}
%\end{itemize}
%\end{frame}
%
%\begin{frame}[fragile]{Sample ADMM implementation: lasso}
%\begin{verbatim}
%prox_f = @(v,rho) (rho/(1 + rho))*(v - b) + b;
%prox_g = @(v,rho) (max(0, v - 1/rho) - max(0, -v - 1/rho));
%
%AA = A*A';
%L  = chol(eye(m) + AA);
%
%for iter = 1:MAX_ITER
%    xx = prox_g(xz - xt, rho);
%    yx = prox_f(yz - yt, rho);
%
%    yz = L \ (L' \ (A*(xx + xt) + AA*(yx + yt)));
%    xz = xx + xt + A'*(yx + yt - yz);
%  
%    xt = xt + xx - xz;
%    yt = yt + yx - yz;
%end
%\end{verbatim}
%\end{frame}
%
%\begin{frame}{Figure}
%\begin{center}
%\psfrag{k}[t][b]{$k$}
%\psfrag{fbest - fmin}[b][t]{$f_\mathrm{best}^{(k)} - f^\star$}
%\psfrag{noise-free realize}{noise-free case}
%\psfrag{realize1}{realization 1}
%\psfrag{realize2}{realization 2}
%\includegraphics[height=0.8\textheight]{figures/pwl_error_fbest_realize.eps}
%\end{center}
%\end{frame}
%
%\begin{frame}{Algorithm}
%    if $L$ is not known (usually the case), can use the following line search:
%    \noindent\rule[-5pt]{\textwidth}{0.4pt}
%    {\footnotesize
%    \begin{tabbing}
%        {\bf given} $x^k$, $\lambda^{k-1}$, and parameter $\beta \in (0,1)$. \\*[\smallskipamount]
%        Let $\lambda := \lambda^{k-1}$. \\*[\smallskipamount]
%        {\bf repeat} \\
%        \qquad \= 1.\ Let $z := \prox_{\lambda g}(x^{k} - \lambda \nabla f(x^{k}))$. \\
%        \> 2.\ {\bf break if} $f(z) \leq \hat{f}_{\lambda}(z, x^{k})$. \\
%        \> 3.\ Update $\lambda := \beta \lambda$. \\*[\smallskipamount]
%        {\bf return} $\lambda^{k} := \lambda$, $x^{k+1}:=z$.
%    \end{tabbing}}
%    \noindent\rule[10pt]{\textwidth}{0.4pt}
%
%    typical value of $\beta$ is $1/2$, and 
%    \[
%    \hat{f}_\lambda(x,y) = f(y) + \nabla f(y)^T (x - y) + 
%    (1/2\lambda)\|x - y\|_2^2
%    \]
%\end{frame}


\end{document}
